We define a linear automata as 
a tuple $(Q, \Delta, q_0, v_0, F, i, o, c)$.
Where $Q$ is a set of states.
$q_0$ is the initial state.
$v_0$ is the initial register values.
$F$ is a field.
$i$ is the size of input vectors,
$o$: output vectors,
$c$: constraints.
$\Delta$ is a set of tuples denoting transitions $(s, G_{c, i+r}, G_c, O_{o, i+r}, O_o, U_{r, r}, U_r, s')$.

If in state $s$, 
and register vector $v$,
with input $i$,
we can traverse a transition $(s, G_m , G_v, O_m, O_v, U_m, U_v, s')$ if $G_m(v \lozenge i) \leq G_v$ (here, $\lozenge$ denotes vertical concatenation).
This outputs $O_m(v) + O_v$ as the output vector, updates the registers to $U_mv + U_v$, and takes us to state $s'$.
This process continues to end of input
or till no edge accepts out of current state.