\begin{definition}[Variable Set]
A variable set is a set of labels $V$. A variable set can be interpreted with an interpretation function, taking labels to integers: $\llbracket \cdot \rrbracket : V \to \mathbb{Z}$.
\end{definition}

\begin{definition}[Linear Relation Set]
A linear relation set is a $4-$tuple $(L, O, C, R)$,
where $L$, $C$, and $R$ are variable sets,
and where $O : (\B)^{\Z \times \Z}$ is a set of binary operators over the integers to the booleans.
It is defined as the following set of expressions:
$$\{l \oplus r + c : l \in L, \oplus \in O, c \in C, r \in R\}$$
\end{definition}

\begin{definition}[Transition]
A transition is considered over four variable sets:
\begin{itemize}
\item $I$, the input variable set, which are the set of fields in each input token
\item $O$, the output variable set, which are the set of fields in each output token
\item $R$, the register set, which is the set of registers
\item $C$, the constant set, a set of background constants
\end{itemize}
These variable labels are used in three linear relation sets:
\begin{itemize}
    \item $\phi = I \times \{\le, \ge\} \times R \times C$, which intuitively is the set describing the guard of the transition.
    \item $\chi = R'\times \{=\}\times R\times C$, which intuitively is the set describing the update to the registers, with $R'$ being the same set as $R$, but updated versions.
    \item $\chi = R'\times \{=\}\times O\times C$, which intuitively is the set describing the output values. 
\end{itemize}
The three sets are used in transition denoted as follows:
$$ p \xrightarrow{\phi / \chi / \psi} q$$
Such that if in state $p$ and $\phi$ is true, we go to state $q$ and require $\chi \wedge \psi$.
\end{definition}

\begin{definition}[$w$-machine]
A $w$-machine is given by a tuple:
$(I, O, R, r_0, Q, q_0, \Delta, F)$
where $I$ is the set of input variable labels,
$O$ is the set of output variable labels,
$R$ is the set of register variable labels,
$r_0 : R \to \mathbb{Z}$ takes registers to their initial values,
$Q$ is the set of states,
$q_0$ is the initial state,
$F \subseteq Q$ is the set of final states,
and transitions as defined in the previous given by $\Delta \subseteq Q \times \phi \times \chi \times \psi \times Q$.
Provided the actions defined for transitions (including the update of registers),
the execution of the machine is what you would expect from a transducer.
\end{definition}