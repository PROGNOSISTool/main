Big counter automata are a refinement on the previous that tightly contain the languages we're considering.
They are regular multi-counter automata, 
but with the ability to take non-deteriministic positive and negative steps:
this simulates many things we see in the processes that we are learning.
Examples of this are a time counter increasing or non-sequential, but increasing, packet numbers.

\begin{definition}
Big step automata is a $X-$machine
with data $X = \mathbb{N}^* \times I \times O$ 
-- a finite number of counters with an input and output.
The input and output are both a finite list of numbers,
but possible different sizes.
We discuss the $i$-th counter with $X_i$.
We discuss the $j$-th input value for state $x$ with $X_I_j$,
and similarly with output: $X_O_j$
A transition in a BSA from any two states $c$ to $d$
must be a relation $(x, x')$ on $X$ definable with a conjunction of
some of the following formulae for any $i$ and $j$:
$x'_i = x_i + \{-1, 0, 1\}$ or 
$x'_i >/< x_i$ or 
$x'_O_j = x_i$ or
$x'_O_j = c$ (a constant) or
$x_I_j =/</> x_i$.
Also, the domain of the relation must be defineable with
some convex interval of the counters.
And, the domains of the relations from some node must be disjoint.
\end{definition}

