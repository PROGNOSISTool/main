

\tikzstyle{module} = [rectangle, draw,
    text width=5em, text centered, rounded corners, 
    minimum height=4em, node distance=5cm]
\tikzstyle{close_module} = [rectangle, draw,
    text width=5em, text centered, rounded corners, 
    minimum height=4em, node distance=2cm]

\tikzstyle{line} = [draw, -latex']


\begin{figure}
    \centering
    \begin{tikzpicture}[node distance = 2cm, auto]
        % Place nodes
        \node [module] (sul) {System under learning};
        \node [module, right of=sul] (l*) {Learn Lib};
        \node [module, right of=l*] (synth) {Synthesizer};
        \node [close_module, below of=l*] (driver) {Driver};
        \node [close_module, below of=synth] (problemer) {Problem Finder};
        
        \path [line] (l*) edge [bend left = 20] node {m and e queries}(sul);
        \path [line] (sul) edge [bend left = 20] node {counter-examples}(l*);
        \path [line] (l*) -- node {NFA}(synth);
        \path [line] (driver) edge [bend right = 0] node {Exploit goal}(problemer);
        \path [line] (synth) edge node {Synthesized model}(problemer);
        \path [line] (synth) edge [bend right = 60] node {Membership queries}(sul);
    \end{tikzpicture}
    \caption{
    Between the system under learning and learn lib is the familiar MAT learning model.
    Once learned, an non-deterministic finite automata model of the SUL is passed to the synthesier.
    The CEGIS synthesizer then tries to synthesize a refinement of the NFA to a mealy machine with registers, using random walks of synthesized mealy machines to see if they are accepted by system under learning.
    Combined with an an exploit goal from the driver (e.g. explode memory), the problem finder tries to find a trace that achieves the exploit or show proof otherwise.
    }
\end{figure}