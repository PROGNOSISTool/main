Counter is a language that accepts a series of ones, 
increasing a register for each one,
and outputting zero.
When a zero is passed, 
the register value is output, and then reset.
This is a simple example of the kind of automata the we seek to learn.

In theory, though impossible,
we want to learn an entire Mealy machine for this language.
This machine still has traditional DFA edges,
but has one state for each value of the counter and an infinite output alphabet.

We can learn, however, some neighborhood of the the states about the initial state.
If we learn a big enough neighborhood, we'll see that a lat of the graph is compressible.
A graph is compressible if many of its nodes have similar neighborhoods.
For example,
the counter language will have many neighborhoods of the shape:
a transition from some node by 1,
a transition to some node by 1,
and transition to the initial node by some number.
What we want to do is take an mealy machine learned by the learner
that approximates the conceptual infinite mealy machine by some finite number of states,
and induce a mealy machine with counters that is equivalent to the infinite mealy machine.

\begin{definition}{Synthesis problem}
Suppose the learner has learnt a mealy machine \mathcal{M}.
And this machine is an approximation for the SUL \mathcal{S}.
We seek to create a big step automata \mathcal{B}
which is equivalent to \mathcal{M},
using induced patterns from \mathcal{S}.
\end{definition}


